\documentclass[11pt]{article}
\usepackage{amsmath,amssymb,graphicx,subcaption}
\usepackage[margin=1in]{geometry}
\usepackage{microtype}
\usepackage{hyperref}

\title{The Polarisation Threshold: Figures and Notes}
\author{BeliefSim.jl replication package}
\date{\today}

\begin{document}
\maketitle

\section{Micro dynamics: OU with stochastic resets}
Figure~\ref{fig:ou} illustrates the individual belief process with Poisson resets. The left panel shows a sample trajectory with reset markers and bands at $\pm \Theta$; the right panel compares the stationary density with and without resets.

\begin{figure}[t]
    \centering
    \begin{subfigure}[b]{0.48\textwidth}
        \includegraphics[width=\textwidth]{figs/fig1_ou_resets_path.pdf}
        \caption{Single-agent OU with stochastic resets (baseline $\lambda=1,\,\sigma=0.8,\,\Theta=2,\,c_0=0.5$).}
    \end{subfigure}\hfill
    \begin{subfigure}[b]{0.48\textwidth}
        \includegraphics[width=\textwidth]{figs/fig1_ou_resets_density.pdf}
        \caption{Stationary density $\bar\rho(x)$ with (red) and without (blue) resets.}
    \end{subfigure}
    \caption{Single-agent dynamics and dispersion. Horizontal lines mark the reset band $\pm\Theta$; red markers denote jumps to $c_0 x_{t^-}$.}
    \label{fig:ou}
\end{figure}

\section{Spectral threshold and pitchfork}
The stability threshold $\kappa^*$ is defined by the spectral condition $\lambda_1(\kappa^*) = 0$, where $\lambda_1$ is the leading odd eigenvalue of the linearised generator. Figure~\ref{fig:eigen} shows $\lambda_1(\kappa)$ and its zero crossing; Figure~\ref{fig:pitchfork} reports the resulting supercritical pitchfork for the order parameter $a(\kappa)$.

\begin{figure}[t]
    \centering
    \includegraphics[width=0.7\textwidth]{figs/fig2_eigen_kappa.pdf}
    \caption{Leading odd eigenvalue $\lambda_1(\kappa)$. The vertical line highlights the spectral threshold $\kappa^*$ where stability changes sign.}
    \label{fig:eigen}
\end{figure}

\begin{figure}[t]
    \centering
    \includegraphics[width=0.75\textwidth]{figs/fig3_bifurcation.pdf}
    \caption{Supercritical pitchfork for the polarization amplitude $a^*(\kappa)$. The dashed branch at $a=0$ loses stability at $\kappa^*$; coloured curves report simulated polarized states.}
    \label{fig:pitchfork}
\end{figure}

\section{Welfare and the externality wedge}
Figure~\ref{fig:welfare} compares decentralised and planner welfare across coupling strengths. The planner internalises fragility from polarization, leading to a lower optimal exposure $\kappa^{\text{soc}}$ than the decentralised choice $\kappa^{\text{dec}}$.

\begin{figure}[t]
    \centering
    \includegraphics[width=0.75\textwidth]{figs/fig4_welfare.pdf}
    \caption{Decentralised versus planner welfare. Vertical markers show $\kappa^{\text{dec}}$ and $\kappa^{\text{soc}}$; the shaded band emphasises the externality wedge.}
    \label{fig:welfare}
\end{figure}

\section{Note on $\kappa^*$ heuristics}
The operational definition of the polarisation threshold is spectral: $\kappa^*$ solves $\lambda_1(\kappa^*)=0$. Earlier back-of-the-envelope shortcuts of the form $\kappa^*\propto 1/V^*$ are retained only as OU-style heuristics and are not generally valid once stochastic resets and state-dependent hazards are present.

\end{document}
